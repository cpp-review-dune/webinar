\section{El DUNE verso: módulos}

\begin{frame}
	\frametitle{\secname}
	\framesubtitle{\subsecname}

	\begin{columns}
		\begin{column}{0.5\textwidth}
			\dirtree{%
				.1 dune-fem.
				.2 dune-grid.
				.3 dune-geometry.
				.4 dune-common.
			}

			\

			\

			\dirtree{%
				.1 dumux.
				.2 dune-istl.
				.2 dune-localfunctions.
				.2 vc.
				.2 psurface.
				.2 superlu.
				.2 arpack++.
				.2 suitesparse.
				.2 dune-alugrid.
				.2 dune-subgrid.
				.2 fmt.
				.2 opm-common.
			}
		\end{column}

		\begin{column}{0.5\textwidth}
			\dirtree{%
				.1 opm-upscaling.
				.2 opm-grid.
				.3 opm-common.
				.4 dune-grid.
				.5 dune-geometry.
				.4 dune-istl.
				.4 boost.
			}

			\

			\

			\dirtree{%
				.1 opm-models.
				.2 opm-material.
				.3 opm-common.
				.4 dune-grid.
				.5 dune-geometry.
				.4 dune-istl.
				.4 boost.
			}
		\end{column}
	\end{columns}

\end{frame}
\subsection{Dependencias de algunos módulos}
\begin{frame}
	\frametitle{\secname}
	\framesubtitle{\subsecname}

	\begin{figure}[ht!]
		\centering
		\includegraphics[width=14.6cm]{dependences}
	\end{figure}

	\begin{description}
		\item[dune-common]

			Clases fundamentales e infraestructura para la construcción del sistema.

			\note{contiene las clases base usadas por todos los módulos de DUNE-modules. Provee algunas clases de infraestructura para depuración y manejo de excepciones así como una librería para manejar una matriz densa.
				and vectors.}

		\item[dune-geometry] Elementos de referencia, métodos de cuadratura y transformaciones geométricas.

		\item[dune-grid] Interfaces con las mallas (ALUGrid, UGGrid, Alberta, YasGrid), construcción y visualización.

		\item[dune-istl] Biblioteca de solucionadores iterativas de plantillas, clases genéricas de matrices/vectores dispersos, solucionadores

		\item[dune-localfunctions]

			Interface genérica para funciones de elementos finitos.
	\end{description}
\end{frame}
