\section{Curso de DUNE/PDELab $2021$}

\begin{frame}[fragile]
	\frametitle{\secname}
	\framesubtitle{\url{https://dune-pdelab-course.readthedocs.io}}

	\begin{figure}[ht!]
		\centering
		\href{https://dune-pdelab-course.readthedocs.io}{\includegraphics[height=8cm]{dune_course_2021}}
	\end{figure}
	\note{
		Cursos de DUNE anuales, lista de correo de preguntas y noticias.

		Cuenta con nueve tutoriales.
		El módulo se llama dune-pdelab-tutorials, \url{https://gitlab.dune-project.org/pdelab/dune-pdelab-tutorials}, es el sucesor de dune-grid-howto \url{https://gitlab.dune-project.org/core/dune-grid-howto}
	}
\end{frame}

\begin{frame}[fragile]
	\frametitle{Snippet en C++}
	\lstinputlisting[caption={Programa \texttt{dune-basics.cc}.},label=dune-basics.cc,]{dune-basics.cc}
	\note{
		Se muestra un código minimo de un programa basado en Dune.
		La forma de trabajo es importar las clases con las directivas %\lstinline|#include <dune/modulo/cabecera.hh>|

		\url{https://www.dune-project.org/sphinx/content/sphinx/dune-fem/mcf_nb.html}
		Es una simulación dinámica.
	}
\end{frame}

{
\usebackgroundtemplate{
	\centering
	\includegraphics[width=\paperwidth]{tutorial-9}
}

\begin{frame}[plain]
	\note{
		Presentar el vídeo presenta la simulación de un flujo que obedece las ecuaciones de Navier Stokes, alrededor de un cilindro.
	}
\end{frame}
}
\begin{frame}[fragile]
	\frametitle{Snippet en Python}
	\begin{figure}[ht!]
		\centering
		\href{https://www.dune-project.org/sphinx/content/sphinx/dune-fem/mcf_nb.html}{\includegraphics[height=8cm]{python_code}}
	\end{figure}

\end{frame}

{
\usebackgroundtemplate{\centering\includegraphics[width=\paperwidth]{jupyter01}}
\begin{frame}[plain]
	\note{
		Mostrar el artículo de dune-Python \url{https://arxiv.org/abs/1807.05252}, página 11.
		Los bindings de dune se encuentran en cada módulo, por ejemplo, dune-common, dune-geometry, dune-grid, dune-istl, etc, que depende de la instalación en C++ de dicho módulo.
	}
\end{frame}
}