\input{dune-webinar-preamble}

\begin{document}

{
\usebackgroundtemplate{\centering\includegraphics[width=\paperwidth]{duneslides-logo}}
\begin{frame}[plain]
	\frametitle{\secname}
\end{frame}
}

\frame[plain,noframenumbering]{\titlepage}

\section{Introducción/Presentación del libro}

\begin{frame}
	\frametitle{\secname}
	A.
\end{frame}

\section{DUNE Numerics Project}
\subsection{Introducción}

\begin{frame}
	\frametitle{\secname}
	\framesubtitle{\subsecname}

	una caja de herramientas modular para la solución numérica de EDPs

	\begin{table}[ht!]
		\caption{Comparación de la línea de comando del gestión de software (fuente: \url{wiki.archlinux.org})}
		\centering\footnotesize
		\begin{tabular}{cccp{50pt}cc}
			\toprule
			Acción             & Arch                    & Red Hat/Fedora          & Debian/Ubuntu                                          & SLES/openSUSE           & Gentoo
			\tabularnewline
			\midrule
			Instala paquetes   & \lstinline|pacman -S|  & \lstinline|dnf install|  & \lstinline|apt install|                                 & \lstinline|zypper install|  & \lstinline|emerge -a|
			\tabularnewline
			Elimina paquetes   & \lstinline|pacman -Rs|  & \lstinline|dnf remove|  & \lstinline|apt remove|                                 & \lstinline|zypper remove|  & \lstinline|emerge -C|
			\tabularnewline
			Busca paquetes     & \lstinline|pacman -Ss| & \lstinline|dnf search| & \lstinline|apt search|                                & \lstinline|zypper search| & \lstinline|emerge -S|
			\tabularnewline
			Actualiza paquetes & \lstinline|pacman -Syu| & \lstinline|dnf upgrade| & \lstinline|apt update \&\&|\newline\lstinline|apt upgrade| & \lstinline|zypper update| & \lstinline|emerge -u world|
			\tabularnewline
			\bottomrule
		\end{tabular}
	\end{table}

\end{frame}

\subsection{Introducción}

\begin{frame}
	\frametitle{\secname}
	\framesubtitle{\subsecname}

	\begin{description}
		\item[dune-common]
			Es minimalista.
			Se usa herramienta pequeñas que sigue la filosofía de UNIX, de modo que tengas una base muy pequeña que deje configurar la máquina de manera más cómoda para el usuario.
		\item[dune-geometry]
			Mantener la paquetería lo más actualizada posible sin sacrificar la estabilidad. Cuenta con los compiladores más actuales de \lstinline{gcc}, \lstinline{lualatex}, \lstinline{go}, etc.
	\end{description}

\end{frame}

{
\usebackgroundtemplate{\centering\includegraphics[width=\paperwidth]{tutorial-9}}
\begin{frame}[plain]
\end{frame}
}

\begin{frame}
	\frametitle{\secname}
	\framesubtitle{\subsecname}

	En esta ocasión hemos elegido Arch Linux como ambiente de trabajo porque tiene disponible una gran variedad de módulos de DUNE.
	Es recomendable habilitar el repositorio \href{https://wiki.archlinux.org/title/Unofficial\_user\_repositories\#arch4edu}{\texttt{arch4edu}}\footnote{Administrado por Jingbei Li de la Universidad de Tsinghua.}.
\end{frame}

\begin{frame}
	\frametitle{Usando el emulador de terminal}
	% Una consola es un display del sistema
	% En todo lo que sigue, usaremos el shell \lstinline{bash}.

	% GNU General Public License v2.0
	% GNU Lesser General Public License v2.1
	% BSD 3-Clause "New" or "Revised" License

	% El terminal te da acceso al shell, que nos da acceso al sistema operativo subyacente.
	% El shell, sabe interactuar entre el usuario y el sistema operativo.

	% CLI Command Line Interface
\end{frame}

\begin{frame}
	% pwd
	% El cursor parpadeante al final de la línea es donde se ingresan los comandos basados ​​en texto. Puede comenzar a experimentar con un comando simple, Imprimir directorio de trabajo (pwd), que muestra la carpeta actual en la que se encuentra en la pantalla. Escriba pwd y presione Entrar.

	% Todos los comandos que ingresa funcionan de la misma manera: ingresa el comando, incluye cualquier parámetro para extender el uso del comando y presiona Enter para ejecutar la línea de comando que ingresó. Escribe en la Terminal: uname -a y presiona Enter. Esto muestra información del sistema con respecto a Mint
\end{frame}

\begin{frame}[fragile]
	\begin{lstlisting}
    gitpod ~/dune-basics $ cd
  \end{lstlisting}
\end{frame}

%\lstinputpath{../../../sandbox}

% \begin{frame}[fragile]
% 	\frametitle{Classes}

% 	\lstinputlisting[
% 		caption={Programa \texttt{hello-linux.cc}.},
% 		label=hello-linux.cc,
% 	]{../../../sandbox/hello-linux.cc}


% \end{frame}

% \begin{frame}[fragile]
% 	\frametitle{Classes}

% 	\lstinputlisting[
% 		caption={Programa \texttt{dune-basics.cc}.},
% 		label=dune-basics.cc,
% 	]{../../../src/dune-basics.cc}

% \end{frame}

% \begin{frame}[fragile]
% 	\frametitle{Classes}

% 	\lstinputlisting[
% 		caption={Programa \texttt{dune-basics.cc}.},
% 		label=dune-basics.cc,
% 	]{../../../sandbox/dune-math-constants.cc}

% \end{frame}
%https://archlinux.org/logos/archlinux-icon-crystal-256.svg

\begin{frame}
	\frametitle{El comando \lstinline{duneproject}}
	% https://gitlab.dune-project.org/core/dune-common/-/raw/master/bin/duneproject
	Es un asistente en el lenguaje \lstinline{bash} que se encuentra en \lstinline{/usr/bin/duneproject}\ldots

\end{frame}

\begin{frame}\transblindsvertical
	\frametitle{Referencias}
	%------------------------------------------------------------ 1
	\only<1>{
		\begin{itemize}
			\item Libros
			      \nocite{*}
			      \printbibliography[heading=none,keyword=book]
		\end{itemize}
	}
	%------------------------------------------------------------ 2
	\only<2>{
		\begin{itemize}
			\item Artículos
			      \printbibliography[heading=none,keyword=paper]
		\end{itemize}
	}
	%------------------------------------------------------------ 3
	\only<3>{
		\begin{itemize}
			\item Sitios web
			      \printbibliography[heading=none,keyword=online]
		\end{itemize}
	}
\end{frame}

\end{document}
%https://github.com/easybuilders/easybuild/wiki/EasyBuild-Tech-Talks-I%3A-Open-MPI
% https://www.youtube.com/playlist?list=PLagFkXs2BczZO4B_I6YhuS4xOxlSPgZuT
% https://www.youtube.com/playlist?list=PLagFkXs2BczZGsvnN7UAqd5LWfgOrat0v
% https://www.youtube.com/playlist?list=PLZRRlbOTxTmAMASXs7mAFnKkFZ3ohefTF
% https://www.youtube.com/playlist?list=PLZRRlbOTxTmBERH5Ov1kWhPlmOQAP04zi
% https://www.youtube.com/playlist?list=PLZRRlbOTxTmCaBYeLHQyqMFyI6caKsVDd
% https://www.youtube.com/playlist?list=PLZRRlbOTxTmB_gV8rLQ6N9KDX5po5eTay
% https://www.youtube.com/playlist?list=PLZRRlbOTxTmCyLkmjWg0e8cNk6rONT-mk