\input{dune-webinar-preamble}

\begin{document}

{
\usebackgroundtemplate{
	\centering
	\includegraphics[width=\paperwidth]{duneslides-logo}
}
\begin{frame}[plain,noframenumbering]

	\color{c++reviewduneblue}

	\begin{flushleft}\bfseries\scshape\huge
		Una introducción a la caja de herramientas DUNE Numerics
		para la solución de modelos matemáticos
	\end{flushleft}

	\

	\

	\

	\

	\

	\

	\begin{minipage}{0.47\textwidth}
		\begin{figure}[ht!]
			\centering
			\includegraphics[height=1.5cm]{alfaomega}
			\caption*{
				\large
				\bfseries
				\textcolor{c++reviewduneblue}{Webinar 12 de Julio de 2021}
			}
		\end{figure}
	\end{minipage}
	\begin{minipage}{0.5\textwidth}
		\begin{flushright}
			\large
			\bfseries
			Elaborado por:\\
			John Jairo Leal Gómez\\
			Universidad Nacional de Colombia\\
			Carlos Alonso Aznarán Laos\\
			Universidad Nacional de Ingeniería, Perú
		\end{flushright}
	\end{minipage}
\end{frame}
}
\section{Presentación del libro}

\begin{frame}
	\frametitle{\secname}
	\begin{minipage}{0.47\textwidth}
		\begin{figure}[ht!]
			\centering
			\includegraphics[height=7.7cm]{portada}
		\end{figure}
	\end{minipage}
	\begin{minipage}{0.5\textwidth}
		\textsc{\Large Capítulos:}

		\

		\begin{enumerate}
			\item

			      Introducción a los números reales $\mathbb{R}$.

			      \

			\item

			      Introducción a las funciones.

			      \

			\item

			      La derivada.

			      \

			\item

			      Modelamiento matemático.

			      \

			\item
			      Anexos.
		\end{enumerate}
	\end{minipage}

\end{frame}

\begin{frame}
	\frametitle{\secname}
	\begin{figure}[ht!]
		\centering
		\includegraphics[height=7.8cm]{vida_cotidiana}
	\end{figure}
\end{frame}

\section{DUNE Numerics Project}
\subsection{Introducción}

\begin{frame}
	\frametitle{\secname}
	\framesubtitle{\subsecname}

	\begin{alertblock}{Distributed and Unified Numerics Environment (DUNE)}
		\note{
			Se hará una breve presentación de la caja de herramientas modular DUNE, biblioteca modular desarrollada en la universidad de Heildeberg Alemania en C++ y Python, para resolver ecuaciones diferenciales parciales utilizando métodos basados en mallas, por ejemplo diferencias finitas, elementos finitos o volúmenes finitos.
		}
		\begin{itemize}
			\item

			      Software de código abierto bajo la licencia GNU General Public Licence 2~\gpllicense{}.

			\item

			      Disponible en
			      \href{https://github.com/dune-copasi/homebrew-tap}{macOS}, \href{https://packages.debian.org/search?suite=sid&section=all&arch=any&searchon=sourcenames&keywords=dune-}{Debian}~\debian{},
			      \href{https://launchpad.net/~opm/+archive/ubuntu/ppa}{Ubuntu}~\ubuntu{},
			      \href{https://build.opensuse.org/search?search_text=dune-&search_for=2&name=1&attrib_type_id=}{openSUSE}~\opensuse{},
			      \href{https://aur.archlinux.org/packages/?O=0&SeB=n&K=dune-&outdated=&SB=n&SO=a&PP=50&do_Search=Ir}{Arch Linux}~\archlinux{} y FreeBSD~\freebsd{}.

			      \note{
				      Desarrollado con CMake, escrito en C++ con enlaces Python a través de pybind11.
			      }

			\item

			      Conjunto de bibliotecas C++ con enlaces a Python.

			\item

			      Utilizado en la resolución de ecuaciones diferenciales parciales e implementación de métodos basados en mallas, por ejemplo diferencias finitas, elementos finitos o volúmenes finitos.
		\end{itemize}
		\note{
			Se mostrará la estructura general, proyectos basados en DUNE y algunas simulaciones de modelos matemáticos que incluyen éste tipo de ecuaciones, así como una implementación breve de DUNE.
		}
	\end{alertblock}

	\begin{figure}[ht!]
		\centering
		\includegraphics[height=4.2cm]{dunedesign}
		\caption{Tomado de \url{https://dune-project.org}.}
	\end{figure}

\end{frame}

\begin{frame}
	\frametitle{\secname}
	\framesubtitle{\subsecname}

	\begin{columns}
		\begin{column}{0.5\textwidth}
			\begin{alertblock}{Proyectos que emplean DUNE}
				\begin{itemize}
					\item \url{https://dune-project.org/about/dune}
					\item \url{https://dumux.org}
					\item \url{https://opm-project.org}
					\item \url{https://www.zib.de/projects/kaskade7-finite-element-toolbox}
				\end{itemize}
			\end{alertblock}
		\end{column}

		\begin{column}{0.5\textwidth}
			\begin{figure}[ht!]
				\centering
				\includegraphics[width=7.5cm]{blood_girke}
				\caption{Tomado de \url{https://dune-project.org}.}
			\end{figure}
		\end{column}
	\end{columns}
\end{frame}

\section{Curso de DUNE/PDELab $2021$}

\begin{frame}[fragile]
	\frametitle{\secname}
	\framesubtitle{\url{https://dune-pdelab-course.readthedocs.io}}

	\begin{figure}[ht!]
		\centering
		\includegraphics[height=8cm]{dune_course_2021}
	\end{figure}
	\note{Cursos de DUNE anuales, lista de correo de preguntas y noticias}
\end{frame}

\begin{frame}[fragile]
	\frametitle{\secname}
	\framesubtitle{Snippet en C++}
	\lstinputlisting[caption={Programa \texttt{dune-basics.cc}.},label=dune-basics.cc,]{dune-basics.cc}
\end{frame}

\begin{frame}[fragile]
	\frametitle{\secname}
	\framesubtitle{Snippet en Python}
	\lstinputlisting[caption={Programa \texttt{dune-basics.cc}.},label=dune-basics.cc,]{dune-basics.cc}
\end{frame}

{
\usebackgroundtemplate{\centering\includegraphics[width=\paperwidth]{tutorial-9}}
\begin{frame}[plain]
\end{frame}
}

{
\usebackgroundtemplate{\centering\includegraphics[width=\paperwidth]{jupyter01}}
\begin{frame}[plain]
\end{frame}
}

%\section{DUNE Latinoamérica}

{
\usebackgroundtemplate{\centering\includegraphics[width=\paperwidth]{cpp_review}}
\begin{frame}[plain]
\end{frame}
}

%\framesubtitle{Página para revisión de DUNE}
%\framesubtitle{Imágenes Docker en Achlinux}

{
\usebackgroundtemplate{\centering\includegraphics[width=\paperwidth]{archiso}}
\begin{frame}[plain]
\end{frame}
}
\subsection{Módulos}

\begin{frame}
	\frametitle{\secname}
	\framesubtitle{\subsecname}

	\begin{columns}
		\begin{column}{0.5\textwidth}
			\dirtree{%
				.1 dune-fem.
				.2 dune-grid.
				.3 dune-geometry.
				.4 dune-common.
			}

			\

			\

			\dirtree{%
				.1 dumux.
				.2 dune-istl.
				.2 dune-localfunctions.
				.2 vc.
				.2 psurface.
				.2 superlu.
				.2 arpack++.
				.2 suitesparse.
				.2 dune-alugrid.
				.2 dune-subgrid.
				.2 fmt.
				.2 opm-common.
			}
		\end{column}

		\begin{column}{0.5\textwidth}
			\dirtree{%
				.1 opm-upscaling.
				.2 opm-grid.
				.3 opm-common.
				.4 dune-grid.
				.5 dune-geometry.
				.4 dune-istl.
				.4 boost.
			}

			\

			\

			\dirtree{%
				.1 opm-models.
				.2 opm-material.
				.3 opm-common.
				.4 dune-grid.
				.5 dune-geometry.
				.4 dune-istl.
				.4 boost.
			}
		\end{column}
	\end{columns}

\end{frame}
\subsection{El DUNE verso: módulos}
\subsection{Dependencias de algunos módulos}
\begin{frame}
	\frametitle{\secname}
	\framesubtitle{\subsecname}

	\begin{figure}[ht!]
		\centering
		\includegraphics[width=14.6cm]{dependences}
	\end{figure}

	\begin{description}
		\item[dune-common]

			Clases fundamentales e infraestructura para la construcción del sistema.

			\note{contiene las clases base usadas por todos los módulos de DUNE-modules. Provee algunas clases de infraestructura para depuración y manejo de excepciones así como una librería para manejar una matriz densa.
				and vectors.}

		\item[dune-geometry] Elementos de referencia, métodos de cuadratura y transformaciones geométricas.

		\item[dune-grid] Interfaces con las mallas (ALUGrid, UGGrid, Alberta, YasGrid), construcción y visualización.

		\item[dune-istl] Biblioteca de solucionadores iterativas de plantillas, clases genéricas de matrices/vectores dispersos, solucionadores

		\item[dune-localfunctions]

			Interface genérica para funciones de elementos finitos.
	\end{description}
\end{frame}

\begin{frame}\transblindsvertical
	\frametitle{Referencias}
	%------------------------------------------------------------ 1
	\only<1>{
		\begin{itemize}
			\item Libros
			      \nocite{*}
			      \printbibliography[heading=none,keyword=book]
		\end{itemize}
	}
	%------------------------------------------------------------ 2
	\only<2>{
		\begin{itemize}
			\item Artículos
			      \printbibliography[heading=none,keyword=paper]
		\end{itemize}
	}
	%------------------------------------------------------------ 3
	\only<3>{
		\begin{itemize}
			\item Sitios web
			      \printbibliography[heading=none,keyword=online]
		\end{itemize}
	}
\end{frame}

\end{document}
%https://conan.iwr.uni-heidelberg.de/events/dune-course_2021/