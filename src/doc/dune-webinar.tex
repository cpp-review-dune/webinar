\input{dune-webinar-preamble}

\begin{document}

{
\usebackgroundtemplate{\centering\includegraphics[width=\paperwidth]{duneslides-logo}}
\begin{frame}[plain,noframenumbering]

	\color{c++reviewduneblue}

	\begin{flushleft}\bfseries\scshape\huge
		Una introducción a la caja de herramientas Dune en
		C++/Python para la solución de modelos matemáticos
	\end{flushleft}

	\

	\

	\

	\

	\

	\

	\begin{minipage}{0.47\textwidth}
		\begin{figure}[ht!]
			\centering
			\includegraphics[height=1.5cm]{alfaomega}
			\caption*{\large\bfseries\textcolor{c++reviewduneblue}{Webinar Julio de 2021}}
		\end{figure}
	\end{minipage}
	\begin{minipage}{0.5\textwidth}
		\begin{flushright}\large\bfseries
			Elaborado:\\
			John Jairo Leal Gómez\\
			Universidad Nacional de Colombia\\
			Carlos Alonso Aznarán Laos\\
			Universidad Nacional de Ingeniería, Perú
		\end{flushright}
	\end{minipage}

\end{frame}
}

%\frame[plain,noframenumbering]{\titlepage}

\section{Presentación del libro}

\begin{frame}
	\frametitle{\secname}%\hspace{-1cm}
	\begin{minipage}{0.47\textwidth}
		\begin{figure}[ht!]
			\centering
			\includegraphics[height=7.7cm]{portada}
		\end{figure}
	\end{minipage}%\hspace{1cm}
	\begin{minipage}{0.5\textwidth}
		\textsc{\large Capítulos:}
		\begin{enumerate}
			\item Introducción a los números reales $\mathbb{R}$
			\item Introducción a las funciones
			\item La derivada
			\item Modelamiento matemático
			\item Anexos
		\end{enumerate}
	\end{minipage}

\end{frame}

\begin{frame}
	\frametitle{\secname}

	\begin{figure}[ht!]
		\centering
		\includegraphics[height=7.7cm]{vida_cotidiana}
	\end{figure}

\end{frame}

\section{DUNE Numerics Project}
\subsection{Introducción}

\begin{frame}
	\frametitle{\secname}
	\framesubtitle{\subsecname}

	\begin{alertblock}{Distributed and Unified Numerics Environment (DUNE)}
		Conjunto de bibliotecas C++/Python para la implementación de métodos basados en mallas, por ejemplo diferencias finitas, elementos finitos o volúmenes finitos.
		De código abierto bajo la licencia GNU General Public Licence 2.
		Los binarios están en las distribuciones Linux Debian, Ubuntu (\href{https://launchpad.net/~opm/+archive/ubuntu/ppa}{PPA}) y OpenSUSE; y las recetas de compilación en freeBSD, Arch Linux.

		\

		Proyectos que emplean DUNE:
		\begin{itemize}
			\item \url{https://dune-project.org/about/dune}
			\item \url{https://dumux.org}
			\item \url{https://opm-project.org}
			\item \url{https://www.zib.de/projects/kaskade7-finite-element-toolbox}
		\end{itemize}
	\end{alertblock}

	\begin{figure}[ht!]
		\centering
		\includegraphics[height=3cm]{dunedesign}
	\end{figure}

\end{frame}

\subsection{Módulos}

\begin{frame}
	\frametitle{\secname}
	\framesubtitle{\subsecname}

	\begin{columns}
		\begin{column}{0.5\textwidth}
			\dirtree{%
				.1 dune-fem.
				.2 dune-grid.
				.3 dune-geometry.
				.4 dune-common.
			}

			\

			\dirtree{%
				.1 dumux.
				.2 dune-istl.
				.2 dune-localfunctions.
				.2 vc.
				.2 psurface.
				.2 superlu.
				.2 arpack++.
				.2 suitesparse.
				.2 dune-alugrid.
				.2 dune-subgrid.
				.2 fmt.
				.2 opm-common.
			}
		\end{column}

		\begin{column}{0.5\textwidth}
			\dirtree{%
				.1 opm-upscaling.
				.2 opm-grid.
				.3 opm-common.
				.4 dune-grid.
				.5 dune-geometry.
				.4 dune-istl.
				.4 boost.
			}

			\

			\dirtree{%
				.1 opm-models.
				.2 opm-material.
				.3 opm-common.
				.4 dune-grid.
				.5 dune-geometry.
				.4 dune-istl.
				.4 boost.
			}
		\end{column}
	\end{columns}

\end{frame}
\subsection{El DUNE verso: módulos}
\begin{frame}
	\frametitle{\secname}
	\framesubtitle{\subsecname}

	\begin{figure}[ht!]
		\centering
		\includegraphics[width=14.5cm]{dependences}
	\end{figure}

	\begin{description}
		\item[dune-common]

		\item[dune-geometry]

		\item[dune-grid]

		\item[dune-istl]

		\item[dune-localfunctions]

			%\item[dune-functions]

	\end{description}

\end{frame}

{
\usebackgroundtemplate{\centering\includegraphics[width=\paperwidth]{tutorial-9}}
\begin{frame}[plain]
\end{frame}
}

\begin{frame}
	\frametitle{\secname}
	\framesubtitle{\subsecname}
\end{frame}

%\begin{frame}[fragile]
%	\begin{lstlisting}
%    gitpod ~/dune-basics $ cd
%  \end{lstlisting}
%\end{frame}


\begin{frame}[fragile]
	\frametitle{\lstinline{dune-python}}

\begin{python}
from dune.common import FieldVector
\end{python}

\end{frame}

\begin{frame}\transblindsvertical
	\frametitle{Referencias}
	%------------------------------------------------------------ 1
	\only<1>{
		\begin{itemize}
			\item Libros
			      \nocite{*}
			      \printbibliography[heading=none,keyword=book]
		\end{itemize}
	}
	%------------------------------------------------------------ 2
	\only<2>{
		\begin{itemize}
			\item Artículos
			      \printbibliography[heading=none,keyword=paper]
		\end{itemize}
	}
	%------------------------------------------------------------ 3
	\only<3>{
		\begin{itemize}
			\item Sitios web
			      \printbibliography[heading=none,keyword=online]
		\end{itemize}
	}
\end{frame}

\end{document}